\subsection{Potential earnings from the platform}

%here 

Finally, we investigate whether there is any indication of social inequality when it comes to how the hosts of different background are treated, and how they price their listings. More specifically, we analyze social inequality in two ways:  i) what price do hosts from different  gender and ethnical backgrounds ask for; ii) whether the demographics of the host or their super-host status play a part in the ratings they receive from their guests.

Logically, both price and ratings are highly influenced by the type of property (number of rooms) and experience (e.g., amenities that are available on the site such as Wi-Fi or a hot tub) that is offered to the guests. To control for this dimension we conduct our analysis only for single private room places in \ab. %We also control for the location demand by computing median of the desired variables in each tract. We describe each analysis next:

Starting with pricing, for each tract we compute the average price based on all the private room properties that fall within that tract. We then compute a metric referred to as the \emph{price residual} by dividing the price of each listing by the average private room price for that tract. A value of less that one for this metric would indicate that the host is under-selling their property compared to the average room in the same area. We then compute a multiple linear regression model with the price residual as the dependent variable  and age, gender,  race and the super-host status as the independent variables. The model suggests a negative association between the \aam \ ethnicity and the price residual. A one-unit increase on the listings by \aam hosts predicts a decrease  of 0.12  in the  price residual  (se = 0.05); this decrease is significant, t(643 ) = -2.141 , p $<$ 0.01.  This result suggests that African-American AirBnB users earn 12\% less rent than other hosts for the same type of house in the same type of location. We do not find any  associations between the rest of the independent variables, which also suggests that the super-hosts do not over/under-sell their property. 


To understand whether there is a racial bias in how the hosts are rated, we use the location rating of the hosts as it corresponds to the satisfaction of the  guests with the location of the property and is meant to be independent of other factors such as amenities available at the property.  Similar to the previous metric we calculate the \emph{location rating residual} which corresponds to the    location rating of a listing divided by the average location rating of the tract.  A value of less that one for this metric indicates that the hosts are unfairly scored down. We then compute a multiple linear regression model with the location rating residual as the dependent variable  and age, gender,  race and the super-host status as  the independent variables. The model suggests a positive association between the super-host status and the location rating residual. That is a one-unit in the number of super-hosts   predicts a slight increase  of 0.01  in the  location rating residual  (se = 0.003); this increase is significant, t(643 ) = -2.755 , p $<$ 0.01.    We do not find any  associations between the rest of the independent variables, which also suggests that based on our dataset we do not observe that hosts of different demographic and racial background get scored down systematically. However, our result suggests that the super-hosts have an slight advantage and they are scored higher in terms of location rating compared to others hosting the same neighborhood.  


%super hosts and eldely get slightly higher rating for the location
