The issue of social inequality in the broader geospatial socio-technical space has been examined extensively in the past.  For instance, in the field of Volunteered Geographical Information (VGI), which includes projects such as OpenStreetMap, Haklay et al. reported that areas with higher deprivation levels were also more likely to suffer from lack of mapping coverage~\cite{haklay2010good}.   Mashhadi et al.\cite{mashhadi2013putting} found that both socio-economic factors (e.g., income deprivation) and physical distance from the city centre are negatively correlated with OpenStreetMap coverage in London, UK.   Hecht et al. ~\cite{hecht2014tale} showed that there is a geographical bias (towards urban areas) in adoption rates, quantity and quality of information in VGI.  Quattrone et al. showed that there is a strong culture bias in editing behavior in VGI and that countries with lower Power Distance are more likely to contribute to such platforms~\cite{quattrone2015there,quattrone2014mind}. 
	
 
Examining social inequality in the geospatial socio-technical literature with specific focus on \emph{tangible} assets, Quattrone et al \cite{quattrone16} found that Airbnb listings of London have increased over time to cover poorer and less educated neighborhoods but these offerings did not attract as many guests. Similarly, Thebault-Spieker et al. \cite{Thebault-Spieker17} compared the relative effectiveness of two sharing economy platforms, UberX and TaskRabbit from a geographical perspective. They showed that because of the correlation between low social-economic neighborhoods and ethnical minority, these neighborhoods suffer from identical lower sharing economy effectiveness. 

In contrast to these works, Fraiberger et al. argued that sharing economy platforms (in particular ride-sharing) benefit mostly poorer populations~\cite{fraiberger2015peer}. Although their work has been criticized because of its reliance on simulation, the economic modeling of their data makes it a valuable paper that cannot be dismissed. Kooti et al. studied the UberX ride-sharing platform by collecting source and destination of the rides from Uber receipts that are sent at the end of each trip to 4.1 million riders and 222 thousand drivers who have Yahoo Mail account~\cite{kooti2017analyzing}. They argued that Uber is not an \emph{all-serve-all} market, rather the riders have higher income than drivers and differ along racial and gender groups.


Other works have taken a qualitative approach to examine the impact of the sharing economy on inequality, but often limitations in scale. Schor et al. \cite{schor2017does} has claimed an increasing inequality within the bottom 80\% of users of sharing economy platforms. Their study is based on qualitative interviews of 43 earners of three platforms (Airbnb, RelayRides and TaskRabbit) from which they conclude the following two reasons for this disparity: first the well-off and highly educated providers are using the platforms to increase their earnings. Second this group is doing work that is traditionally done by people of low educational status. Similarly, Edelman et al.  showed the impact of race as a factor of inequality by showing that users with stereotypically African American names in Airbnb are less likely to be accepted as guests compared to identical profiles with stereotypically white names~\cite{edelman2017racial}. 

%The closest work to ours is a recent report publisehd in BrooklyDeep.org which claims across all 72 predominantly Black New
%York City neighborhoods, Airbnb hosts are 5 times more likely to be white.  In those neighborhoods, the Airbnb host
%population is 74\% white, while the white resident population is only 13.9\%. ?White Airbnb hosts in Black neighborhoods earned in total $159.7 million, compared to only $48.3 million for Black hosts
%Our results also confirm these findings but additionally we show that when it comes to earning  the disparity is caused by lack of 

%There is also some anecdotal evidence that users are disproportionately white\footnote{http://brooklyndeep.org/report-airbnb-as-racial-gentrification-tool/}. Yet we still do not fully understand what do hosts with different ethnicity background offer and what part they play in the sharing economy platforms.

Unlike previous works, we aim here to build a more complete picture of service providers on sharing economy platforms, by leveraging a more diverse range of data points: in particular, census data, demographic data, and aesthetic analysis. By leveraging deep learning technology, we can also achieve this at a much larger scale allowing us to draw additional insights into the cost and benefits of these platforms. 

% ADD