In this paper we studied the demographics of  \ab \  hosts who listed their properties in the greater Chicago area during March 2017, to understand the impact of  social inequality in the sharing economy platform.  Our results show that listings are typically geographically located in richer and denser areas with respect to median household income, and that minorities are under-represented even in minority-majority areas. Furthermore, we showed that social inequality  manifests itself not only in the lack  of participation of the minorities but also in the way they present their listings visually and the price they ask for.  We showed that  the potential earnings of  African-Americans hosts appear to be 12\% less than that of other hosts for the same type of property in the same location.  However, in our study we did not observe any  unequal treatment of female or elderly hosts.  

\subsection{Implication}  
%summary of results



Documenting and providing information on social inequality  in the sharing economy is only a first step.  As a second step, it is important to know 
how these results can be used to prevent social inequality. Many critics of the sharing economy argue that external regulations posed by authorities is the way forward to prevent social inequality and discrimination. However, we believe there are big opportunities for the sharing economy platforms to assist the under-represented users through \emph{internal policies}.  In a  recent report\footnote{https://blog.atairbnb.com/wp-content/uploads/2016/09/REPORT\_Airbnbs-Work-to-Fight-Discrimination-and-Build-Inclusion.pdf},  \ab \  has suggested withholding information regarding the users as  a means to tackle discrimination against the users from different ethnical backgrounds.  We argue that while this may change usage patterns, 
a potentially  more efficient approach would be to introduce internal policies that could assist hosts by enabling them to present their property in more aesthetically pleasing ways and for a fairer price. \ab \ as a frontier example of sharing economy platforms should aim to bring to the fore  means of leveling the playing field that help under-represented communities increase their visibility on the platform.


\subsection{Limitation}

Our data and so in turn our analysis  has some inherent limitations as we only study  hosts and their listings based on the property  type, price, location and images, and not based on other factors that could impact the success of a host such as  house rules and cancellation policy. Furthermore, we did not look into the linguistics and how the hosts present themselves in terms of both summary and description of their place as well as their  exchanges and interactions with the guests. Nonetheless believe that our findings are a significant contribution to the debate of social inequality in the sharing economy. As part of future direction we believe it is essential to conduct a temporal  study of  social inequality in \ab \ to understand how hosts of different background behave and adapt to the platform overtime. 