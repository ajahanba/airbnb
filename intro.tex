Sharing Economy platforms provide services and connections between individuals with under-utilized tangible assets such as a car or a house, and other individuals or businesses in need of those assets~\cite{FRENKEN20173,sprague2015worker}. In the past years, these platforms have become extremely popular as they offer to increase consumer welfare by opening up competition in an increasingly large variety of domains~\cite{smith2016shared}. Indeed some speculators including Milbourn \cite{milbourn2015future} and Nunberg \cite{nunberg2016goodbye} widely believe that the sharing economy will substantially displace traditional equivalents in the future.

Despite this promise, the sharing economy raises important concerns regarding socio-economic inequality, manifested as age and racial discrimination. A recent study of Uber, the ride sharing platform, showed that African-american passengers were subject to longer waits~\cite{ge2016racial}. Similarly,  a field study of \ab  \ has shown that  guests with African-american names were more likely to be turned down~\cite{edelman2017racial}, criticizing \ab \ for not being an inclusive platform.  In fact, this widespread critic manifested itself on social media where users shared their discriminatory experiences using the hash tag \#Airbnbwhileblack and led to AirBnB's new anti-discrimination policy and internal report to build inclusion~\cite{murphy2016airbnb}. 

However due to lack of data, independent studies of social inequality in sharing economy platforms are still limited and we are yet to understand how race, gender and age affect the users of these platforms. As a result, it can be difficult for effective policies to be derived or to determine the potential effectiveness of such policies. Cue et al. have argued for example that \ab \ suffers from the \emph{statistical} discrimination rather than \emph{taste-based} discrimination~\cite{cui2016discrimination}. Through a field experiment with 1,000 Airbnb hosts, they found that when guests have even one positive review on their profiles, it statistically eliminates racial discrimination against them. They hence suggest that closer examination of the reviewing process should be an important aspect of any policy in this domain. In short, a better understanding of the user population, the behaviors of the hosts and their customers, the socio-economic environment can yield to better and more effective solutions to problems of inequality and discrimination.

In this paper, we are interested in understanding i) who are the hosts on AirBnB, how are they distributed across age, race and gender; ii) what they offer in terms of property (shared, vs entire place) and where these listings are geographically; iii) how they visually present  their property on the platform; iv) and finally what ratings and number of reviews they receive.  To answer these questions we investigate AirBnB listings of Chicago, and detect the ethnicity, age and gender of  2700  hosts. We match  the information found on the Airbnb platform to the US Census data on the census-tract levels in which listings are located. This allows us to study the impact of income on the economic activity of the platform and measure racial disparity. Finally, leveraging advances in machine learning we quantify the aesthetic score of the main photo of the property and study how  hosts from different socio-economic and racial backgrounds present their property on \ab \ platform. 


Our findings show that listings are typically geographically located in richer and denser areas with respect to median household income. We also identify that minorities are under-represented even in minority-majority areas. This discrepancy is further confirmed with respect to the visual presentation of properties and listing prices, where potential earnings of \aam s appear to be 12\% less than that of other hosts.  We aim for our methodology to help organizations both explore issues of inequality and discrimination on sharing economy platforms and discuss various ways in which policies can be put in place to assist.



